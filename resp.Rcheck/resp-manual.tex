\nonstopmode{}
\documentclass[letterpaper]{book}
\usepackage[times,inconsolata,hyper]{Rd}
\usepackage{makeidx}
\usepackage[utf8]{inputenc} % @SET ENCODING@
% \usepackage{graphicx} % @USE GRAPHICX@
\makeindex{}
\begin{document}
\chapter*{}
\begin{center}
{\textbf{\huge Package `resp'}}
\par\bigskip{\large \today}
\end{center}
\begin{description}
\raggedright{}
\inputencoding{utf8}
\item[Title]\AsIs{Creates response models for whole datasets}
\item[Version]\AsIs{1.0.2}
\item[Description]\AsIs{Given a data.frame and selecting fixed and random factors, returns a set of models that allow to select the significant effect of the fixed factors.}
\item[Depends]\AsIs{R (>= 3.0.2)}
\item[Imports]\AsIs{lme4, stats}
\item[Encoding]\AsIs{UTF-8}
\item[Author]\AsIs{Adrià Masip }\email{adria@burgeon.cat}\AsIs{}
\item[Maintainer]\AsIs{resp Team }\email{webmaster@burgeon.cat}\AsIs{}
\item[License]\AsIs{GPL (>= 2)}
\end{description}
\Rdcontents{\R{} topics documented:}
\inputencoding{utf8}
\HeaderA{resp-package}{Creates response models for whole datasets}{resp.Rdash.package}
\aliasA{resp}{resp-package}{resp}
\keyword{package}{resp-package}
%
\begin{Description}\relax
Given a data.frame and selecting fixed and random factors, returns a set of models that allow to select the significant effect of the fixed factors.
\end{Description}
%
\begin{Details}\relax

The DESCRIPTION file:
This package was not yet installed at build time.\\{}

Index:  This package was not yet installed at build time.\\{}
\textasciitilde{}\textasciitilde{} An overview of how to use the package, including the most important functions \textasciitilde{}\textasciitilde{}
\end{Details}
%
\begin{Author}\relax
Adrià Masip <adria@burgeon.cat>

Maintainer: resp Team <webmaster@burgeon.cat>
\end{Author}
%
\begin{References}\relax
\textasciitilde{}\textasciitilde{} Literature or other references for background information \textasciitilde{}\textasciitilde{}
\end{References}
%
\begin{SeeAlso}\relax
\textasciitilde{}\textasciitilde{} Optional links to other man pages, e.g. \textasciitilde{}\textasciitilde{}
\textasciitilde{}\textasciitilde{} \code{\LinkA{<pkg>}{<pkg>}} \textasciitilde{}\textasciitilde{}
\end{SeeAlso}
%
\begin{Examples}
\begin{ExampleCode}
~~ simple examples of the most important functions ~~
\end{ExampleCode}
\end{Examples}
\inputencoding{utf8}
\HeaderA{formula\_from\_vec}{Formula string creation from \code{vector} elements}{formula.Rul.from.Rul.vec}
%
\begin{Description}\relax
Given a vector, create a unique string from the elements.
\end{Description}
%
\begin{Usage}
\begin{verbatim}
formula_from_vec(x, start='~ ', mid=' + ', end='', as_formula=FALSE)
\end{verbatim}
\end{Usage}
%
\begin{Arguments}
\begin{ldescription}
\item[\code{x}] 
\code{vector} to be transformed

\item[\code{start}] 
initial string, DEFAULT: \code{'\textasciitilde{} '}

\item[\code{mid}] 
between-elements string, DEFAULT: \code{' + '}

\item[\code{end}] 
final string, DEFAULT: \code{''}

\item[\code{as\_formula}] 
\code{logical}, return string as S formula, DEFAULT: \code{FALSE}

\end{ldescription}
\end{Arguments}
%
\begin{Value}
Returns single string of concatenated elements
\end{Value}
\inputencoding{utf8}
\HeaderA{log.dataset}{log-transform data.frame}{log.dataset}
%
\begin{Description}\relax
given a whole dataset in data.frame format, log-transform the values avoiding NAs and zeros,
so no -Inf or NaN is created
\end{Description}
%
\begin{Usage}
\begin{verbatim}
log.dataset(data, columns = 1:dim(data)[2])
\end{verbatim}
\end{Usage}
%
\begin{Arguments}
\begin{ldescription}
\item[\code{data}] 
data.frame to be transformed

\item[\code{columns}] 
columns to be transformed, DEFAULT: whole data.frame

\end{ldescription}
\end{Arguments}
%
\begin{Value}
Returns the same dataframe with log-transformed values
\end{Value}
\inputencoding{utf8}
\HeaderA{mod.check}{check models to recover performance information}{mod.check}
%
\begin{Description}\relax
After a \code{mresp} object is created, check for fixed factors significance and unvalidated results,
mod.check can also be used to compare the performance of the created mixed models
\end{Description}
%
\begin{Usage}
\begin{verbatim}
mod.check(models,omit_NA=TRUE)
\end{verbatim}
\end{Usage}
%
\begin{Arguments}
\begin{ldescription}
\item[\code{models}] 
\code{mresp} object to be analyzed

\item[\code{omit\_NA}] 
\code{logical}, compare data without NA values, DEFAULT: TRUE

\end{ldescription}
\end{Arguments}
%
\begin{Value}
Returns the given \code{mresp} object with new \code{check\_out} element inside each response variable
\end{Value}
\inputencoding{utf8}
\HeaderA{mod.resp}{Create response lmer models for a whole data.frame}{mod.resp}
%
\begin{Description}\relax
Given the fixed and random factors of a data.frame, creates a list of 5 mixed models for each response variable,
so the best fit can be selected
\end{Description}
%
\begin{Usage}
\begin{verbatim}
mod.resp(data, fixed, random, r_group, exclude,
 omit_NA = TRUE, fixed_interaction = TRUE, check_models = TRUE)
\end{verbatim}
\end{Usage}
%
\begin{Arguments}
\begin{ldescription}
\item[\code{data}] 
\code{data.frame} to be analyzed

\item[\code{fixed}] 
\code{vector} of column names to be used as fixed factors

\item[\code{random}] 
\code{vector} of column names to be used as random factors

\item[\code{r\_group}] 
\code{vector} of column names to be used as random grouping factors

\item[\code{exclude}] 
\code{vector} of column names to be excluded from the response analysis

\item[\code{omit\_NA}] 
\code{logical}, DEFAULT \code{TRUE}, avoid using NAs from \code{data}

\item[\code{fixed\_interaction}] 
\code{logical}, DEFAULT \code{TRUE}, check interaction from \code{fixed} factors

\item[\code{check\_models}] 
\code{logical}, DEFAULT \code{TRUE}, create \code{check\_out} table for each response variable

\item[\code{lmer\_warnings}] 
\code{logical}, DEFAULT \code{FALSE}, display \code{lmer()} construction warnings

\end{ldescription}
\end{Arguments}
%
\begin{Value}
Returns \code{mresp} object with a list of response variables sorted alphabetically,
with 5 models each and a comparision between them (\code{checkout})

\begin{ldescription}
\item[\code{resp\_1}] First response variable
\item[\code{resp\_...}] Other response variables
\item[\code{resp\_n}] Last response variable
\end{ldescription}

\end{Value}
%
\begin{References}\relax
\code{lmer}, \code{stats}
\end{References}
%
\begin{Examples}
\begin{ExampleCode}
n <- 500L
tr <- c('T1','T2','T3')
sp <- c('S1','S2','S3')
gr <- c('A','B')
F1 <- c();F2 <- c();F3 <- c()
R1 <- c();R2 <- c();R3 <- c()
for (i in 1:n) {
 F1[i] <- tr[round(i/n*3,0)]
 F2[i] <- sp[round(runif(1L,1L,3L),0)]
 F3[i] <- gr[round(runif(1L,1L,2L),0)]
 R1[i] <- rnorm(1,10,2)+runif(1L)+(which(tr==F1[i])*3)
 R2[i] <- rnorm(1,10,2)+runif(1L)+(which(sp==F2[i])*10)
 R3[i] <- rnorm(1,600,20)+runif(1L)+(which(tr==F1[i])*20)
}
table(F1,F2)

x <- data.frame(
    Treatment=F1,
    Specie=F2,
    Group=F3,
    Rand=runif(n),
    Heigth=R1,
    Diameter=R2,
    Number_leaves=R3,
    other=runif(n)
)
# rm(n,tr,sp,gr,F1,F2,F3)
a <- mod.resp(data = x, fixed = c('Treatment','Specie'), random='Rand',
              r_group = c('Group'),exclude='other',lmer_warnings=TRUE,
              choose_models=FALSE,check_models=FALSE)
print(a)
\end{ExampleCode}
\end{Examples}
\inputencoding{utf8}
\HeaderA{name\_range}{Numeric data categorization between user-given ranges}{name.Rul.range}
%
\begin{Description}\relax
Returns 'High', ['Mid[\_n]'] or 'Low' based on range values for group limits
\end{Description}
%
\begin{Usage}
\begin{verbatim}
name_range(x, range)
\end{verbatim}
\end{Usage}
%
\begin{Arguments}
\begin{ldescription}
\item[\code{x}] 
\code{numeric} set to be transformed, must be \code{vector} or \code{data.frame}

\item[\code{range}] 
limits of the groups to be created


\end{ldescription}
\end{Arguments}
%
\begin{Value}
Returns \code{vector} or \code{data.frame} of transformed elements.
\end{Value}
\inputencoding{utf8}
\HeaderA{print.mresp}{Print method for \code{mresp} objects}{print.mresp}
%
\begin{Description}\relax
Prints a resume table of the relations between models based on logLikelihood ratio
\end{Description}
\printindex{}
\end{document}
